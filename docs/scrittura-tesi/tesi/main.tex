% !TeX spellcheck = it_IT
\documentclass{TesiDiaUniroma3}
\usepackage[utf8]{inputenc}
\usepackage{graphicx}
\usepackage[bottom]{footmisc}
\usepackage{xurl}

\titolo{\textbf{Ottimizzazione di Architetture Neurali Compatte con Meccanismi di Attenzione: MobileNetECA per la Classificazione su CIFAR-10}}
\autore{Dmytro Kozak}
\matricola{538593}
\relatore{Prof. Giuseppe Sansonetti}
%\correlatore{} % modifica anche TesiDiaUniroma3.cls se vuoi avere un correlatore
\annoAccademico{2024/2025}

% Pacchetti di utilità
\usepackage[plainpages=false, hidelinks]{hyperref}
\usepackage[all]{hypcap}
\usepackage{amsthm}
\usepackage{amsmath}
\usepackage{graphicx}
\usepackage{float}
\usepackage{hyperref}
\usepackage{amsfonts}
\usepackage{mathtools}
\usepackage[width=.75\textwidth]{caption}
\usepackage{adjustbox}
\usepackage[normalem]{ulem}
\usepackage{makecell}
\usepackage{listings}
\usepackage[table]{xcolor}
\usepackage{tabularx}
\usepackage{booktabs}
\usepackage{subcaption}
\useunder{\uline}{\ul}{}

% Configurazione listings per codice Python
\lstset{
    language=Python,
    basicstyle=\ttfamily\small,
    keywordstyle=\color{blue},
    commentstyle=\color{gray},
    stringstyle=\color{red},
    showstringspaces=false,
    breaklines=true,
    numbers=left,
    numberstyle=\tiny\color{gray},
    frame=single
}

\begin{document}

% Pagine di frontespizio (numerate in romano)
\frontmatter
\generaFrontespizio
\introduzione{introduzione}
\generaIndice
\generaIndiceFigure
\generaIndiceTabelle

% Pagine di tesi (numerate in arabo)
\mainmatter
\capitolo{Introduzione e Contesto}{capitolo1}
\capitolo{Fondamenti Teorici}{capitolo2}
\capitolo{Architettura MobileNetECA}{capitolo3}
\capitolo{Metodologia Sperimentale}{capitolo4}
\capitolo{Risultati e Analisi}{capitolo5}
\capitolo{Discussione}{capitolo6}

% Parte finale della tesi
\backmatter
\conclusioni{conclusioni}
\ringraziamenti{ringraziamenti}

% Bibliografia con BibTeX
\bibliography{bibliografia}

\end{document}
