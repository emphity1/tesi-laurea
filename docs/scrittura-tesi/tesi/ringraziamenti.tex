%\section*{Ringraziamenti}

Al termine di questo percorso di tesi triennale, desidero esprimere la mia gratitudine a tutte le persone che hanno contribuito, direttamente o indirettamente, alla realizzazione di questo lavoro.

\vspace{0.5cm}

Ringrazio innanzitutto il \textbf{Prof. Giuseppe Sansonetti}, mio relatore, per avermi guidato in questo progetto e per avermi introdotto al mondo delle reti neurali efficienti e dell'edge computing. La sua disponibilità, i suoi consigli e la libertà che mi ha concesso nell'esplorazione del tema sono stati fondamentali per la mia crescita accademica.

\vspace{0.5cm}

Un ringrazi

amento particolare va alla \textbf{comunità open source} e ai ricercatori che hanno reso disponibili implementazioni, dataset e benchmark che hanno costituito la base di questo lavoro. In particolare, ringrazio gli autori di MobileNetV2, ECA-Net e PyTorch per aver condiviso le loro innovazioni con la comunità scientifica.

\vspace{0.5cm}

Ringrazio i miei \textbf{compagni di corso} per i momenti di confronto, supporto reciproco e collaborazione durante questi tre anni. Le discussioni, i dubbi condivisi e le sessioni di studio insieme hanno reso questo percorso più ricco e stimolante.

\vspace{0.5cm}

Un ringraziamento sincero va alla mia \textbf{famiglia}, per il supporto costante, la pazienza e l'incoraggiamento nei momenti di difficoltà. Senza il loro sostegno, questo traguardo non sarebbe stato possibile.

\vspace{0.5cm}

Infine, dedico questo lavoro a chi crede che la tecnologia debba essere accessibile, sostenibile ed etica. Spero che modelli compatti come MobileNetECA possano contribuire a democratizzare l'intelligenza artificiale, portandola su dispositivi di tutti, ovunque nel mondo.

\vspace{1cm}

\begin{flushright}
\textit{Dmytro Kozak} \\
Roma, {DATA\_DISCUSSIONE}
\end{flushright}
