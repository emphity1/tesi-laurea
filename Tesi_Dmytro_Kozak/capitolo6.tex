
In questo capitolo vengono presentati e discussi i risultati sperimentali ottenuti. L'analisi è suddivisa in tre parti:
\begin{itemize}
    \item \textbf{Ablation Study}: Valutazione dell'impatto di ogni singola componente (ECA, Reparameterization, Augmentation) sulle prestazioni.
    \item \textbf{Distillazione della Conoscenza}: Spinta finale delle prestazioni tramite Knowledge Distillation ed Exponential Moving Average sui pesi, per raggiungere i risultati di SOTA.
    \item \textbf{Analisi Dettagliata del Modello Finale}: Studio degli errori tramite matrice di confusione e curve ROC.
    \item \textbf{Confronto con lo Stato dell'Arte}: Posizionamento del nostro modello rispetto ad altre architetture note in letteratura in termini di efficienza (Pareto Frontier).
\end{itemize}

\section{Ablation Study: Impatto delle Componenti}
Per isolare il contributo di ciascuna tecnica proposta, abbiamo addestrato e valutato quattro varianti del modello, partendo dalla baseline MobileNetV2 fino alla configurazione finale.
L'addestramento di ciascuna variante è stato condotto con iperparametri identici (seed, normalizzazione, learning rate, weight decay) e con una suddivisione corretta del dataset: 45.000 immagini per il training, 5.000 per la validazione (selezione del modello migliore) e 10.000 per il test finale.

\begin{table}[H]
    \centering
    \caption[Risultati dell'Ablation Study.]{Risultati dell'Ablation Study. Ogni riga rappresenta l'aggiunta di una componente rispetto alla precedente. L'accuratezza di test è misurata una sola volta sul modello selezionato tramite early stopping sulla validazione.}
    \label{tab:ablation}
    \resizebox{\textwidth}{!}{
    \begin{tabular}{l l c c c c}
        \toprule
        \textbf{ID} & \textbf{Configurazione} & \textbf{Test Acc (\%)} & \textbf{Val Acc (\%)} & \textbf{Parametri} & \textbf{Note} \\
        \midrule
        A & MobileNetV2-Micro (Baseline) & 91.65\% & 92.66\% & $\sim$78.0k & Optimized Stride/Layers \\
        B & MobileNetECA (A + ECA) & 92.17\% & 92.58\% & $\sim$78.0k & +0.52\% Test Acc \\
        C & MobileNetECA-Rep (B + Rep) & 92.43\% & 93.84\% & $\sim$62.2k$^*$ & +0.26\% Test Acc \\
        D & MobileNetECA-Rep-AdvAug & 93.23\% & 93.80\% & $\sim$62.2k$^*$ & +0.80\% Test Acc \\
        E & + KD \& EMA (Final) & \textbf{93.76\%} & \textbf{94.60\%} & \textbf{62.2k}$^*$ & \textbf{+0.53\% Test Acc} \\
        \bottomrule
    \end{tabular}
    }
    \vspace{0.2em}
    \small{\textsuperscript{*}Parametri in modalità deploy (rami fusi matematicamente). In training: $\sim$84.7k.}
\end{table}

Come si evince dalla Tabella \ref{tab:ablation}, l'introduzione del blocco \textbf{ECA} (Configurazione B) offre un miglioramento a costo quasi nullo in termini di parametri, dimostrando l'importanza di enfatizzare le feature informative nei canali.
La \textbf{ri-parametrizzazione} (Configurazione C) fornisce un ulteriore boost, permettendo al modello di apprendere rappresentazioni più complesse grazie ai rami multipli in training.
Infine, l'\textbf{Advanced Augmentation} (Configurazione D) è determinante per superare la barriera del 93\%, agendo come forte regolarizzatore e prevenendo l'overfitting. L'ulteriore applicazione di \textbf{Knowledge Distillation ed EMA} (Configurazione E) permette di estrarre il massimo potenziale dall'architettura compatta, raggiungendo il picco finale del 93.76\% senza alcuna aggiunta di parametri in inferenza.


\section{Spinta Finale SOTA: Knowledge Distillation ed EMA}
Essendo il modello \textbf{Mo\-bile\-Net\-ECA-Rep-AdvAug} un'architettura ultra-compatta (solo $\sim$62.2k parametri reali calcolati in fase di inferenza/deploy), le tecniche di regolarizzazione standard come Label Smoothing e Mixup hanno mostrato risultati controproducenti, penalizzando l'accuratezza in quanto il modello è già severamente limitato in termini di capacità memorizzativa. Per superare questa barriera e spingere ulteriormente le prestazioni senza aumentare il costo computazionale in fase di inferenza, abbiamo adottato una strategia basata sulla \textbf{Knowledge Distillation (KD)} combinata con l'\textbf{Exponential Moving Average (EMA)} sui pesi.

Per la Knowledge Distillation, abbiamo impiegato come "Teacher" il modello \textbf{RepVGG-A0}, addestrato ex novo (\textit{from scratch}) nell'ambito di questo lavoro sul medesimo dataset CIFAR-10, fino a fargli raggiungere un'accuratezza robusta del \textbf{94.19\%}. Il modello "Studente" (la nostra architettura) è stato poi addestrato per minimizzare una funzione di perdita composita, definita come una combinazione lineare tra la classica Cross-Entropy rispetto alle etichette reali (Hard Loss) e la Kullback-Leibler (KL) Divergence rispetto alle probabilità "ammorbidite" generate dal Teacher (Soft Loss):

\begin{itemize}
    \item \textbf{Temperatura ($T$)}: Impostata a $T = 4.0$ per "sfumare" le distribuzioni di probabilità del Teacher, consentendo allo Studente di apprendere le relazioni intrinseche tra classi simili (es. "questo cane ha forti tratti associabili anche alla classe gatto").
    \item \textbf{Peso KD ($\alpha$)}: Fissato a $\alpha = 0.5$, assegnando identica importanza alla Hard Loss e alla Soft Loss.
\end{itemize}

Contestualmente, durante l'addestramento è stato mantenuto un modello ombra tramite EMA, i cui pesi venivano aggiornati ad ogni passo computando una media mobile esponenziale con un tasso di decadimento (\textit{decay}) pari a $0.999$. Questo meccanismo funge da regolarizzatore senza parametri, stabilizzando le traiettorie di ottimizzazione. I pesi EMA sono stati utilizzati per la valutazione sul Test Set.

Come evidenziato nella Tabella \ref{tab:kd_results}, l'impiego sinergico di Knowledge Distillation ed EMA ha garantito un guadagno netto di \textbf{+0.53\%}, consentendo al modello di battere architetture significativamente più complesse (come ResNet-32, ferma a 93.53\%) mantenendo inalterato il costo computazionale a \textit{runtime}.

\begin{table}[H]
    \centering
    \caption{Impatto della Knowledge Distillation ed EMA sull'architettura proposta}
    \label{tab:kd_results}
    \begin{tabular}{l c c}
        \toprule
        \textbf{Configurazione} & \textbf{Test Accuracy} & \textbf{Parametri (Deploy)} \\
        \midrule
        Ours (solo AdvAug) & 93.23\% & $\sim$62.2k \\
        \textbf{Ours (AdvAug + KD + EMA)} & \textbf{93.76\%} & \textbf{$\sim$62.2k} \\
        \bottomrule
    \end{tabular}
\end{table}

In conclusione, \textbf{Mo\-bile\-Net\-ECA-Rep-AdvAug} dimostra che è possibile ottenere prestazioni da "Server-Class" (\textbf{93.76\%}) con un budget di risorse hardware da "Microcontroller-Class" ($\sim$62.2k parametri in deploy), grazie alla potente combinazione architetturale tra depthwise convolution, attenzione ECA, ri-parametrizzazione (che fa crollare i parametri utili post-training), e a tecniche di training avanzate come iterazioni di AutoAugment e Knowledge Distillation.

\subsection{Dinamiche di Addestramento}
Analizzando le curve di training (Figura \ref{fig:accuracy_loss}), notiamo comportamenti distinti.

\begin{figure}[H]
    \centering
    \begin{subfigure}{0.8\textwidth}
        \includegraphics[width=\textwidth]{figure/accuracy_comparison.png}
        \caption{Confronto Accuratezza di Validazione}
    \end{subfigure}
    \hfill
    \begin{subfigure}{0.8\textwidth}
        \includegraphics[width=\textwidth]{figure/loss_comparison.png}
        \caption{Confronto Loss di Training (Smoothed)}
    \end{subfigure}
    \caption[Dinamiche di apprendimento per le diverse configurazioni.]{Dinamiche di apprendimento per le diverse configurazioni. Si noti come la configurazione finale (rossa) converga a un'accuratezza superiore nonostante una loss iniziale più alta dovuta alla difficoltà introdotta dall'augmentation.}
    \label{fig:accuracy_loss}
\end{figure}

Il modello con Advanced Augmentation mostra una crescita dell'accuratezza più lenta nelle prime epoche rispetto alla baseline. Questo è atteso: le immagini "tagliate" o distorte sono più difficili da classificare. Tuttavia, nel lungo periodo (epoche $>150$), questa difficoltà costringe la rete ad apprendere feature più robuste e generalizzabili, portando al traguardo del 93.23\% per la configurazione D (successivamente spinto a 93.76\% grazie all'integrazione di KD ed EMA).

Inoltre, riportiamo in Figura \ref{fig:real_dynamics} le dinamiche di addestramento estratte direttamente dai log sperimentali della configurazione finale. Si osserva chiaramente come l'accuratezza di validazione (arancione) rimanga costantemente sopra quella di training (blu). Questo fenomeno è spiegato dalla forte componente di rumore introdotta dalla Data Augmentation sul set di addestramento, che agisce come un ``prezzo'' da pagare in termini di loss istantanea per ottenere una superiore capacità di generalizzazione sui dati reali del test set.

Quantificando questo effetto: all'epoca finale (200), l'accuratezza di training è \textbf{90.48\%} contro una validazione di \textbf{93.70\%}, con un gap positivo di \textbf{+3.22 punti percentuali}. Il gap medio su tutte le 200 epoche è di +5.93 pp, con un massimo nelle prime fasi dell'addestramento dove l'augmentation ha l'effetto più marcato. Il modello raggiunge la migliore accuratezza di validazione (\textbf{93.80\%}) all'epoca 189. Questo ampio gap positivo (val $>$ train) conferma che l'augmentation agisce come forte regolarizzatore: il modello generalizza \textit{meglio} di quanto memorizzi i campioni di training.

\begin{figure}[H]
    \centering
    \begin{subfigure}{0.48\textwidth}
        \includegraphics[width=\textwidth]{figure/v3_train_vs_val_acc.png}
        \caption{Accuratezza (Training vs Validation)}
    \end{subfigure}
    \hfill
    \begin{subfigure}{0.48\textwidth}
        \includegraphics[width=\textwidth]{figure/v3_train_loss.png}
        \caption{Progressione Loss di Training}
    \end{subfigure}
    \caption[Dinamiche reali dell'addestramento finale.]{Dinamiche reali dell'addestramento finale (200 epoche). I dati evidenziano la stabilità della convergenza grazie allo schedule Cosine Annealing.}
    \label{fig:real_dynamics}
\end{figure}

\section{Analisi Dettagliata del Modello Finale (Configurazione E: KD+EMA)}
Il modello finale è stato sottoposto a un'analisi più approfondita sul Test Set (per una scomposizione chirurgica delle metriche su ogni singola classe, si rimanda all'Appendice~\ref{app:metrics}).

\subsection{Matrice di Confusione}
La matrice di confusione (Figura \ref{fig:confusion_matrix}) ci permette di visualizzare le classi in cui il modello commette più errori.

\begin{figure}[H]
    \centering
    \includegraphics[width=1\textwidth]{figure/confusion_matrix.png}
    \caption{Matrice di Confusione del modello MobileNetECA-Rep-AdvAug sul Test Set.}
    \label{fig:confusion_matrix}
\end{figure}

Osserviamo che la diagonale principale è dominante, indicando un'ottima capacità di classificazione generale. Le confusioni più frequenti avvengono tra classi semanticamente simili:
\begin{itemize}
    \item \textbf{Gatto vs Cane}: Errore classico dovuto alla similarità morfologica (quattro zampe, pelo, orecchie).
    \item \textbf{Camion vs Automobile}: Entrambi veicoli su ruote, spesso confusi se lo sfondo è simile (strada).
\end{itemize}
Tuttavia, il tasso di errore è contenuto, confermando l'efficacia del meccanismo di attenzione nel discriminare dettagli sottili.

\subsection{Curve ROC e AUC}
Le curve ROC (Receiver Operating Characteristic) mostrano il trade-off tra True Positive Rate e False Positive Rate.
Per una migliore leggibilità, riportiamo un ingrandimento (zoom) dell'angolo in alto a sinistra (Figura \ref{fig:roc_zoomed}).

\begin{figure}[H]
    \centering
    \includegraphics[width=1.0\textwidth]{figure/roc_curve_zoomed.png}
    \caption[Dettaglio delle curve ROC per le 10 classi.]{Dettaglio delle curve ROC per le 10 classi (Zoom TPR 0.8-1.0). L'area sotto la curva (AUC) è prossima a 1.0 per tutte le classi, indicando un'eccellente separabilità.}
    \label{fig:roc_zoomed}
\end{figure}

L'AUC medio è estremamente elevato ($>0.99$ per classi facili come "Nave" o "Aereo"), dimostrando che il modello assegna probabilità molto alte alla classe corretta e basse alle altre.

\section{Analisi Qualitativa Avanzata}
Oltre alle metriche globali, è fondamentale comprendere \textit{dove} e \textit{perché} il modello prende le sue decisioni. Abbiamo condotto due analisi specifiche: la sensibilità all'occlusione (per capire quali parti dell'immagine sono determinanti) e l'analisi degli errori ad alta confidenza.

\subsection{Occlusion Sensitivity Analysis}
Per verificare che il modello non stia sfruttando shortcut spurie (come il colore dello sfondo), abbiamo analizzato la \textit{Occlusion Sensitivity}. In Figura \ref{fig:occlusion_sensitivity}, mostriamo come varia la confidenza della classe corretta quando oscuriamo diverse parti dell'immagine con un patch scorrevole.
\begin{figure}[H]
    \centering
    \includegraphics[width=1.0\textwidth]{figure/occlusion_sensitivity.png}
    \caption[Mappe di Sensibilità all'Occlusione.]{Mappe di Sensibilità all'Occlusione. (Riga 1) Immagine originale. (Riga 2) Heatmap: le aree rosse indicano le zone dove l'occlusione causa il maggior calo di confidenza. (Riga 3) Sovrapposizione. Si nota come il modello si concentri correttamente sulle caratteristiche discriminanti dell'oggetto (es. il muso del cane, la carrozzeria dell'auto) ignorando lo sfondo.}
    \label{fig:occlusion_sensitivity}
\end{figure}
Le heatmap confermano che il modulo ECA guida l'attenzione della rete sulle feature morfologiche rilevanti dell'oggetto, ignorando efficacemente il background. Questo comportamento è indice di una buona generalizzazione.

\subsection{Analisi degli "Imperdonabili": High Confidence Errors}
Un aspetto critico per l'affidabilità è capire quando il modello sbaglia con alta certezza ("overconfident").
In Figura \ref{fig:top_errors}, riportiamo alcuni esempi dal Test Set dove il modello ha sbagliato previsione con una confidenza superiore al 90\%.

\begin{figure}[H]
    \centering
    \includegraphics[width=0.85\textwidth]{figure/top_errors.png}
    \caption[Errori ad Alta Confidenza ($>90\%$).]{Errori ad Alta Confidenza ($>90\%$). Ogni immagine riporta la classe vera (True), quella predetta (Pred) e la confidenza. Molti di questi errori sono dovuti ad ambiguità visive estreme (es. prospettive insolite o oggetti parzialmente occlusi) che metterebbero in difficoltà anche un osservatore umano.}
    \label{fig:top_errors}
\end{figure}
L'analisi rivela che molti di questi "errori" sono in realtà casi limite o etichettature discutibili nel dataset stesso (noisy labels), scagionando parzialmente l'architettura.

\section{Confronto Esteso con lo Stato dell'Arte}
Infine, posizioniamo il nostro lavoro nel panorama della letteratura scientifica ultra-lightweight.


\begin{table}[h]
    \centering
    \label{tab:comparison_sota}
    \caption{Confronto con lo Stato dell'Arte su CIFAR-10. Il nostro modello (MobileNetECA-Rep) ottiene risultati competitivi con una frazione dei parametri e dei FLOPs.}
    \begin{tabular}{l l c c c c}
        \toprule
        \textbf{Modello} & \textbf{Tipo} & \textbf{Accuratezza} & \textbf{Parametri} & \textbf{FLOPs} & \textbf{Rif.} \\
        \midrule
        ResNet-20 & CNN & 92.60\% & 0.27M & 40.81M & \cite{he2016deep} \\
        ResNet-32 & CNN & 93.53\% & 0.47M & 69.12M & \cite{he2016deep} \\
        ResNet-44 & CNN & 94.01\% & 0.66M & 97.44M & \cite{he2016deep} \\
        ResNet-56 & CNN & 94.37\% & 0.86M & 125.75M & \cite{he2016deep} \\
        VGG-16 (bn) & CNN & 94.16\% & 15.25M & 313.73M & \cite{simonyan2014very} \\
        MobileNetV2 (0.5x) & Efficient & 92.88\% & 0.70M & 27.97M & \cite{sandler2018mobilenetv2} \\
        MobileNetV2 (1.0x) & Efficient & 93.79\% & 2.24M & 87.98M & \cite{sandler2018mobilenetv2} \\
        ShuffleNetV2 (0.5x) & Efficient & 90.13\% & 0.35M & 10.90M & \cite{ma2018shufflenet} \\
        RepVGG-A0 & Rep & 94.39\% & 7.84M & 489.08M & \cite{ding2021repvgg} \\
        \textbf{MobileNetECA-Rep (Ours)} & \textbf{Hybrid} & \textbf{93.49\%} & \textbf{0.08M} & \textbf{10.73M} & This Work \\
        \bottomrule
    \end{tabular}
\end{table}


Come mostrato nella Tabella \ref{tab:comparison_sota}, MobileNetECA-Rep-AdvAug eccelle nella categoria "sotto i 100k parametri".

In letteratura sono presenti approcci efficienti recenti come GhostNet~\cite{han2020ghostnet}, che raggiunge il 93.38\% con 0.43M parametri applicando il modulo Ghost a ResNet-56, e MobileNetV3-Small~\cite{howard2019searching}, che ottiene il 92.97\% con 2.54M parametri. Il nostro modello si distingue da entrambi per l'ordine di grandezza inferiore di parametri (\textbf{0.06M} vs 0.43M--2.54M), mantenendo un'accuratezza competitiva e persino superiore (93.76\%).

L'analisi qualitativa dei campioni classificati erroneamente rivela due categorie principali di errore:
\begin{enumerate}
    \item \textbf{Background Confusion}: Alcuni \textit{Uccelli} (Bird) vengono classificati come \textit{Aerei} (Plane) se lo sfondo è azzurro uniforme. Questo indica che in alcuni casi il modello fa ancora troppo affidamento sul contesto (colore dello sfondo) piuttosto che sull'oggetto.
    \item \textbf{Fine-grained Features}: La distinzione tra \textit{Automobile} e \textit{Camion} (Truck) fallisce per i pickup, che possiedono caratteristiche ibride (cabina da auto, cassone da camion). La risoluzione $32 \times 32$ penalizza fortemente questo tipo di discriminazione fine-grained.
\end{enumerate}

Questa analisi suggerisce che ulteriori miglioramenti potrebbero derivare da un aumento della risoluzione (es. $64 \times 64$ o $128 \times 128$) o dall'integrazione di meccanismi di attenzione spaziale oltre a quelli di canale (es. CBAM).

\section{Confronto con lo Stato dell'Arte: Efficienza}
Infine, posizioniamo il nostro lavoro nel panorama della letteratura scientifica in termini di Pareto Efficiency.

Come mostrato nella Tabella \ref{tab:comparison_sota} e visivamente nella Figura \ref{fig:efficiency}, il nostro modello occupa una posizione di rilievo.

\begin{figure}[H]
    \centering
    \includegraphics[width=1\textwidth]{figure/efficiency_params.png}
    \caption[Efficiency Frontier: Accuratezza vs Numero di Parametri.]{Efficiency Frontier: Accuratezza vs Numero di Parametri (scala logaritmica). Il nostro modello (stella rossa) si trova nettamente a sinistra (meno parametri) rispetto a modelli con accuratezza simile come ResNet-32 o GhostNet.}
    \label{fig:efficiency}
\end{figure}

\subsection{Discussione}
\textbf{Confronto con ResNet}: Otteniamo performance simili a una ResNet-32/44 ma con una frazione dei parametri ($62k$ vs $\sim500k$).
\subsection{Analisi Empirica della Latenza e Implementabilità}
Un aspetto cruciale per il deployment su dispositivi edge è la latenza di inferenza (riassunta in Tabella~\ref{tab:latency}). Per validare l'efficacia della ri-parametrizzazione, abbiamo misurato le prestazioni del modello finale in modalità "deploy" (con kernel fusi, vedi Sez. 5.1).
Le misurazioni sono state effettuate su CPU \textbf{AMD Ryzen Threadripper 3965WX} (32 Core @ 3.8GHz) utilizzando un batch size unitario ($N=1$) per simulare uno scenario real-time.

\begin{table}[H]
    \centering
    \caption{Latenza di Inferenza su CPU (1000 iterazioni, escluse le prime 50 di warmup)}
    \label{tab:latency}
    \begin{tabular}{l c c}
        \toprule
        \textbf{Metrica} & \textbf{Valore} & \textbf{Note} \\
        \midrule
        Mediana (p50) & \textbf{2.68 ms} & Metrica rappresentativa \\
        Media (escl.\ warmup) & 6.94 ms & $\sigma = 15.38$ ms \\
        p95 & 62.97 ms & Spike periodici OS \\
        Throughput & \textbf{144.0 FPS} & Real-Time ($>30$ FPS) \\
        \bottomrule
    \end{tabular}
\end{table}

\paragraph{Nota Metodologica}
Le prime 50 iterazioni sono state escluse dal calcolo delle statistiche per eliminare l'effetto del warmup (JIT compilation, allocazione cache). La mediana di \textbf{2.68 ms} rappresenta la latenza tipica di una singola inferenza. La deviazione standard elevata e il p95 alto sono causati da spike periodici dovuti allo scheduling del sistema operativo, non dal modello stesso: la distribuzione presenta un cluster compatto intorno al valore mediano con outlier sporadici.

Con un throughput superiore a \textbf{144 FPS} su una CPU generica da workstation, il modello garantisce ampiamente prestazioni real-time senza necessità di acceleratori hardware dedicati. \textbf{Sebbene per questo lavoro non siano stati condotti benchmark fisici diretti su hardware edge}, le ridottissime metriche teoriche (sole 9.4M FLOPs e 62.2k parametri) suggeriscono fortemente che su microcontrollori performanti (es. STM32H7) o processori mobile si possa ragionevolmente stimare un'esecuzione fluida.
L'assenza strutturale di operazioni complesse o esoteriche (come le attivazioni Hard-Swish) è stata pensata proprio per bypassare i colli di bottiglia nel porting verso framework embedded reali come TensorFlow Lite Micro.


\section{Analisi dell'Impatto degli Iperparametri}
Oltre all'architettura, la scelta degli iperparametri di addestramento gioca un ruolo cruciale nel raggiungimento della massima accuratezza, specialmente per reti compatte. Abbiamo condotto una Grid Search analizzando l'interazione tra Learning Rate (LR), Weight Decay (WD) e Width Multiplier ($\alpha$).

\paragraph{Nota Metodologica}
La Grid Search è stata condotta sulla configurazione Baseline (A) senza Advanced Augmentation, raggiungendo un massimo di circa 91.4\%. I risultati intermedi (93.23\%) incorporavano le tecniche di augmentation avanzata descritte nella Sezione 5.4, mentre il risultato finale (93.76\%) è ottenuto tramite le tecniche di distillazione descritte nella sezione successiva.

\subsection{Interazione Learning Rate vs Weight Decay}
La Figura \ref{fig:heatmap_lr_wd} mostra come l'accuratezza vari al variare di LR e WD.
Si nota una regione ottimale (zona chiara) lungo la diagonale, suggerendo che un learning rate aggressivo deve essere bilanciato da una regolarizzazione più forte per prevenire l'esplosione dei pesi.

\begin{figure}[H]
    \centering
    \includegraphics[width=0.9\textwidth]{figure/heatmap_lr_wd.png}
    \caption[Heatmap dell'accuratezza al variare di Learning Rate e Weight Decay.]{Heatmap dell'accuratezza (Validation Accuracy) al variare di Learning Rate e Weight Decay. Si osserva che alti valori di LR richiedono un decay maggiore per stabilizzare l'addestramento.}
    \label{fig:heatmap_lr_wd}
\end{figure}

\subsection{Impatto della Capacità del Modello sulla Regolarizzazione}
Analizzando l'interazione tra la larghezza della rete (Width Multiplier) e gli iperparametri, emergono trend coerenti.
Aumentare la larghezza migliora monotonicamente le prestazioni, ma con ritorni decrescenti oltre $\alpha=0.75$. La nostra scelta di $\alpha=0.5$ rappresenta il punto di ginocchio (knee point) ottimale tra prestazioni e computazione.
Inoltre, analisi estensive hanno confermato una relazione inversa tra capacità del modello e regolarizzazione ottimale: modelli più grandi richiedono un Weight Decay maggiore per prevenire l'overfitting, mentre reti molto sottili, avendo intrinsecamente meno capacità di memorizzazione, beneficiano di un approccio più conservativo.

