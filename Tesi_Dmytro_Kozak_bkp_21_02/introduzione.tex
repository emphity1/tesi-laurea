In questa tesi viene affrontato il problema di classificazione di immagini utilizzando il dataset CIFAR-10.
L'obiettivo è analizzare e confrontare diverse architetture di Deep Learning, in particolare le Reti Neurali Convoluzionali (CNN), per valutare le loro prestazioni in termini di accuratezza e complessità computazionale.

Il lavoro è strutturato come segue:
\begin{itemize}
    \item Il Capitolo 1 introduce i concetti fondamentali del Machine Learning e del Deep Learning.
    \item Il Capitolo 2 descrive il dataset CIFAR-10 e le sue caratteristiche.
    \item Il Capitolo 3 approfondisce le architetture delle Reti Neurali Convoluzionali.
    \item Il Capitolo 4 presenta la metodologia sperimentale e l'implementazione dei modelli.
    \item Il Capitolo 5 discute i risultati ottenuti.
\end{itemize}
