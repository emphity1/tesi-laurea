\documentclass{TesiDiaUniroma3}
\usepackage[utf8]{inputenc}
\usepackage{graphicx}
\usepackage[bottom]{footmisc}
%\usepackage{hyperref}
\usepackage{xurl} % Facoltativo per migliorare il wrapping degli URL

\titolo{\textbf{Modelli di classificazione immagini per CIFAR-10}}
\autore{Dmytro Kozak}
\matricola{538593}
\relatore{Prof. Giuseppe Sansonetti}
%\correlatore{Dott.ssa Elena Gagliardoni} % modifica anche TesiDiaUniroma3.cls se vuoi avere un correlatore
%\tutor{Dott.ssa Elena Gagliardoni}
\annoAccademico{2025/2026}
% dati opzionali
% dati opzionali
% --- FINE dati relativi al template TesiDiaUniroma3

% --- INIZIO richiamo di pacchetti di utilità. Questi sono un esempio e non sono strettamente necessari al modello per la tesi.
\usepackage{amsthm}	% per definizioni e teoremi
\usepackage{amsmath}	% per ``cases'' environment
% \usepackage{graphicx} % Already loaded by class file
\usepackage{float}
\usepackage{amsfonts}
\usepackage{mathtools}
\usepackage[width=.75\textwidth]{caption}
\usepackage{adjustbox}
\usepackage[normalem]{ulem}
\usepackage[]{algorithm2e}
% Packege aggiunti da Simone Deriu
\usepackage{makecell} % Per le tabelle
\usepackage{listings}
\usepackage[table]{xcolor}
\usepackage{tabularx}
\usepackage{subcaption} % Added for subfigures
\useunder{\uline}{\ul}{}
\usepackage[plainpages=false, hidelinks]{hyperref}	% generazione di collegamenti ipertestuali su indice e riferimenti (ALWAYS LAST except for packages that depend on it)
\usepackage{hypcap} % per far si che i link ipertestuali alle immagini puntino all'inizio delle stesse e non alla didascalia sottostante
% --- FINE richiamo di pacchetti di utilità

% Spaziatura interlinea 1.5 (standard tesi)
% Spaziatura interlinea 1.5 (reale)
\linespread{1.5}

\begin{document}
% ----- Pagine di fronespizio, numerate in romano (i,ii,iii,iv...) (obbligatorio)
\frontmatter
\generaFrontespizio
\clearpage
\begin{abstract}
    In questa tesi viene proposta un'architettura ibrida per la classificazione efficiente di immagini su dispositivi edge, basata su MobileNetV2, Attention Mechanisms (ECA) e Structural Reparameterization.
    Il modello proposto, MobileNetECA-Rep, raggiunge un'accuratezza del 93.5\% sul dataset CIFAR-10 utilizzando meno di 80k parametri, superando lo stato dell'arte in termini di efficienza (trade-off Accuracy/FLOPs).
    Vengono analizzati nel dettaglio i contributi delle singole componenti tramite ablation study e viene fornita una caratterizzazione qualitativa degli errori.
    I risultati dimostrano che l'ottimizzazione architetturale mirata può compensare la riduzione drastica della capacità del modello.
\end{abstract}
\clearpage

%%%%%%% \ringraziamenti{ringraziamenti}	% inserisce i ringraziamenti e li prende in questo caso da ringraziamenti.tex
% \introduzione{introduzione}		% inserisce l'introduzione e la prende in questo caso da introduzione.tex
\generaIndice
\generaIndiceFigure


% ----- Pagine di tesi, numerate in arabo (1,2,3,4,...) (obbligatorio)
\mainmatter
% il comando ``capitolo'' ha come parametri:
% 1) il titolo del capitolo
% 2) il nome del file tex (senza estensione) che contiene il capitolo. Tale nome \`e usato anche come label del capitolo
\capitolo{Introduzione}{capitolo1}
\capitolo{Il Dataset CIFAR-10}{capitolo2}
\capitolo{Fondamenti delle Reti Neurali}{capitolo3}
\capitolo{Stato dell'Arte e Tecniche di Efficienza}{capitolo4}
\capitolo{Progettazione e Architettura del Sistema}{capitolo5}
\capitolo{Analisi dei Risultati}{capitolo6}
%\capitolo{Titolo Capitolo 6}{capitolo6}

% Appendice
\appendix
\chapter{Dettagli Metriche di Valutazione}
\label{app:metrics}

In questa appendice (Tabella~\ref{tab:class_report}) riportiamo i valori dettagliati di Precision, Recall e F1-Score per ciascuna delle 10 classi del dataset CIFAR-10, ottenuti dal miglior modello \textit{MobileNetECA-Rep-AdvAug}.

\begin{table}[h]
    \centering
    \caption{Report di Classificazione dettagliato per classe.}
    \label{tab:class_report}
    \begin{tabular}{l c c c c}
        \toprule
        \textbf{Classe} & \textbf{Precision} & \textbf{Recall} & \textbf{F1-Score} & \textbf{Support} \\
        \midrule
        Airplane & 0.94 & 0.94 & 0.94 & 1000 \\
        Automobile & 0.96 & 0.97 & 0.97 & 1000 \\
        Bird & 0.92 & 0.90 & 0.91 & 1000 \\
        Cat & 0.87 & 0.85 & 0.86 & 1000 \\
        Deer & 0.94 & 0.94 & 0.94 & 1000 \\
        Dog & 0.88 & 0.89 & 0.88 & 1000 \\
        Frog & 0.93 & 0.95 & 0.94 & 1000 \\
        Horse & 0.95 & 0.96 & 0.96 & 1000 \\
        Ship & 0.96 & 0.96 & 0.96 & 1000 \\
        Truck & 0.96 & 0.96 & 0.96 & 1000 \\
        \midrule
        \textbf{Accuracy} & & & \textbf{0.938} & 10000 \\
        \textbf{Macro Avg} & 0.93 & 0.93 & 0.93 & 10000 \\
        \textbf{Weighted Avg} & 0.93 & 0.93 & 0.93 & 10000 \\
        \bottomrule
    \end{tabular}
\end{table}


% ----- Parte finale della tesi (obbligatorio)
\backmatter
\conclusioni{conclusioni}
\ringraziamenti{ringraziamenti}

% Bibliografia con BibTeX (obbligatoria)
% Non si deve specificare lo stile della bibliografia
\nocite{*}
\bibliography{bibliografia} % inserisce la bibliografia e la prende in questo caso da bibliografia.bib

\end{document}
